\documentclass[12pt]{article}

\usepackage{tikz} % картинки в tikz
\usepackage{microtype} % свешивание пунктуации
\usepackage{array} % для столбцов фиксированной ширины
\usepackage{comment} % для комментирования целых окружений
\usepackage{indentfirst} % отступ в первом параграфе

\usepackage{sectsty} % для центрирования названий частей
\allsectionsfont{\centering}

\usepackage{amsmath, amssymb, amsthm, amsfonts} % куча стандартных математических плюшек

\usepackage[top=2cm, left=1cm, right=1cm, bottom=2cm]{geometry} % размер текста на странице
\usepackage{lastpage} % чтобы узнать номер последней страницы
 
\usepackage{enumitem} % дополнительные плюшки для списков
%  например \begin{enumerate}[resume] позволяет продолжить нумерацию в новом списке

\usepackage{caption} % подписи к рисункам
\usepackage{hyperref} % гиперссылки
\usepackage{multicol} % текст в несколько столбцов


\usepackage{fancyhdr} % весёлые колонтитулы
\pagestyle{fancy}
\lhead{Введение в машинное обучение, ВШЭ}
\chead{}
\rhead{2022-04-01}
\lfoot{Вариант для наших братанов}
\rfoot{It's gonna be legen… wait for it… dary!}
%\rfoot{Тест}
\renewcommand{\headrulewidth}{0.4pt}
\renewcommand{\footrulewidth}{0.4pt}

\usepackage{ifthen} % для написания условий

\usepackage{todonotes} % для вставки в документ заметок о том, что осталось сделать
% \todo{Здесь надо коэффициенты исправить}
% \missingfigure{Здесь будет Последний день Помпеи}
% \listoftodos --- печатает все поставленные \todo'шки


% более красивые таблицы
\usepackage{booktabs}
% заповеди из докупентации:
% 1. Не используйте вертикальные линни
% 2. Не используйте двойные линии
% 3. Единицы измерения - в шапку таблицы
% 4. Не сокращайте .1 вместо 0.1
% 5. Повторяющееся значение повторяйте, а не говорите "то же"


\usepackage{fontspec}
\usepackage{polyglossia}

\setmainlanguage{russian}
\setotherlanguages{english}

% download "Linux Libertine" fonts:
% http://www.linuxlibertine.org/index.php?id=91&L=1
\setmainfont{Linux Libertine O} % or Helvetica, Arial, Cambria
% why do we need \newfontfamily:
% http://tex.stackexchange.com/questions/91507/
\newfontfamily{\cyrillicfonttt}{Linux Libertine O}

% Математические шрифты 
% Математические шрифты 
\usepackage{unicode-math}     
\setmathfont[math-style=upright]{euler.otf} 

\setmathfont[range={\mathbb, \mathop, \heartsuit, \angle, \smile, \varheartsuit}]{Asana-Math.otf}

\AddEnumerateCounter{\asbuk}{\russian@alph}{щ} % для списков с русскими буквами
\setlist[enumerate, 2]{label=\asbuk*),ref=\asbuk*}


% мои цвета https://www.artlebedev.ru/colors/
\definecolor{titleblue}{rgb}{0.2,0.4,0.6} 
\definecolor{blue}{rgb}{0.2,0.4,0.6} 
\definecolor{red}{rgb}{1,0,0.2} 
\definecolor{green}{rgb}{0,0.6,0} 
\definecolor{purp}{rgb}{0.4,0,0.8} 

% цвета из geogebra 
\definecolor{litebrown}{rgb}{0.6,0.2,0}
\definecolor{darkbrown}{rgb}{0.75,0.75,0.75}

% Гиперссылки
\usepackage{xcolor}   % разные цвета

\usepackage{hyperref}
\hypersetup{
  unicode=true,           % позволяет использовать юникодные символы
  colorlinks=true,        % true - цветные ссылки
  urlcolor=blue,          % цвет ссылки на url
  linkcolor=black,          % внутренние ссылки
  citecolor=green,        % на библиографию
  breaklinks              % если ссылка не умещается в одну строку, разбивать её на две части?
}

% эпиграфы
\usepackage{epigraph}
\setlength\epigraphwidth{.5\textwidth}
\setlength\epigraphrule{0pt}

% Математические операторы первой необходимости:
\DeclareMathOperator{\sgn}{sign}
\DeclareMathOperator*{\argmin}{arg\,min}
\DeclareMathOperator*{\argmax}{arg\,max}
\DeclareMathOperator{\Cov}{Cov}
\DeclareMathOperator{\Var}{Var}
\DeclareMathOperator{\Corr}{Corr}
\DeclareMathOperator{\E}{\mathop{E}}
\DeclareMathOperator{\Med}{Med}
\DeclareMathOperator{\Mod}{Mod}
\DeclareMathOperator*{\plim}{plim}

\DeclareMathOperator{\logloss}{logloss}
\DeclareMathOperator{\softmax}{softmax}

\DeclareMathOperator{\tr}{tr}

% команды пореже
\newcommand{\const}{\mathrm{const}}  % const прямым начертанием
\newcommand{\iid}{\sim i.\,i.\,d.}  % ну вы поняли...
\newcommand{\fr}[2]{\ensuremath{^{#1}/_{#2}}}   % особая дробь
\newcommand{\ind}[1]{\mathbbm{1}_{\{#1\}}} % Индикатор события
\newcommand{\dx}[1]{\,\mathrm{d}#1} % для интеграла: маленький отступ и прямая d

% одеваем шапки на частые штуки
\def \hb{\hat{\beta}}
\def \hs{\hat{s}}
\def \hy{\hat{y}}
\def \hY{\hat{Y}}
\def \he{\hat{\varepsilon}}
\def \hVar{\widehat{\Var}}
\def \hCorr{\widehat{\Corr}}
\def \hCov{\widehat{\Cov}}

% Греческие буквы
\def \a{\alpha}
\def \b{\beta}
\def \t{\tau}
\def \dt{\delta}
\def \e{\varepsilon}
\def \ga{\gamma}
\def \kp{\varkappa}
\def \la{\lambda}
\def \sg{\sigma}
\def \tt{\theta}
\def \Dt{\Delta}
\def \La{\Lambda}
\def \Sg{\Sigma}
\def \Tt{\Theta}
\def \Om{\Omega}
\def \om{\omega}

% Готика
\def \mA{\mathcal{A}}
\def \mB{\mathcal{B}}
\def \mC{\mathcal{C}}
\def \mE{\mathcal{E}}
\def \mF{\mathcal{F}}
\def \mH{\mathcal{H}}
\def \mL{\mathcal{L}}
\def \mN{\mathcal{N}}
\def \mU{\mathcal{U}}
\def \mV{\mathcal{V}}
\def \mW{\mathcal{W}}

% Жирные буквы
\def \mbb{\mathbb}
\def \RR{\mbb R}
\def \NN{\mbb N}
\def \ZZ{\mbb Z}
\def \PP{\mbb{P}}
\def \QQ{\mbb Q}

\def \putyourname{\fbox{
    \begin{minipage}{42em}
      Фамилия, имя, номер группы:\vspace*{3ex}\par
      \noindent\dotfill\vspace{2mm}
    \end{minipage}
  }
}

\def \checktable{

  \vspace{5pt}
  Табличка для проверяющих работу:

\vspace{5pt}

  \begin{tabular}{|m{2cm}|m{1cm}|m{1cm}|m{1cm}|m{1cm}|m{1cm}|m{1cm}|m{1cm}|m{2cm}|}
\toprule
    Вопрос & 11 &  12 & 13 & 14 & 15 & 16 & 17 & Итого \\
\midrule
    Баллы &  &  & & & & & &  \\
 \bottomrule
\end{tabular}
}


\def \testtable{

\vspace{5pt}
  Внесите сюда ответы на тест:

\vspace{5pt}

\begin{tabular}{|m{2cm}|m{0.6cm}|m{0.6cm}|m{0.6cm}|m{0.6cm}|m{0.6cm}|m{0.6cm}|m{0.6cm}|m{0.6cm}|m{0.6cm}|m{0.6cm}|}
\toprule
    Вопрос & 1 &  2 & 3 & 4 & 5 & 6 & 7 & 8 & 9 & 10 \\
\midrule
    Ответ &  &  & & & & & & & & \\
 \bottomrule
\end{tabular}
}


% [1][3] 1 = one argument, 3 = value if missing
% эта магия создаёт окружение answerlist
% именно в окружении answerlist записаны варианты ответов в подключаемых exerciseXX
% просто \begin{answerlist} сделает ответы в три столбца
% если ответы длинные, то надо в них руками сделать
% \begin{answerlist}[1] чтобы они шли в один столбец
\newenvironment{answerlist}[1][3]{
\begin{multicols}{#1}

\begin{enumerate}[label=\fbox{\emph{\Alph*}},ref=\emph{\alph*}]
}
{
\item Нет верного ответа.
\end{enumerate}
\end{multicols}
}

% BB: unicol version. don't know why \ifthenelse fails in second part of new-env
\newenvironment{answerlistu}{
\begin{enumerate}[label=\fbox{\emph{\Alph*}},ref=\emph{\alph*}]
}
{
\item Нет верного ответа.
\end{enumerate}
}


\excludecomment{solution} % without solutions

\theoremstyle{definition}
\newtheorem{question}{Вопрос}

\usepackage{tikzlings}
\usepackage{tikzducks}

\usepackage{alltt}

\begin{document}

\putyourname

\testtable

\checktable

\mbox{ }

\epigraph{Братан никогда не обижается, если другой Братан не перезвонил, или не ответил на сообщение, своевременно.}{\textit{Кодекс Братана: Статья 145}}

\textbf{Это нулевой вариант экзамена. Он нужен для того, чтобы его формат не стал для вас сюрпризом.} Работа состоит из трёх частей: тестовая, задачи и ответы на открытые вопросы. Списывание карается обнулением работы. Удачи!

\section*{Часть первая: тестовая} 

Дайте ответ на $10$ тестовых вопросов. Каждый вопрос стоит $3$ балла. Никакие дополнительные пояснений в этой части работы от вас не требуются.

\begin{question}
Выберите одно верное утверждение про решающие деревья
\begin{answerlist}
  \item  В каждой внутренней вершине дерева проверяется некоторое условие
  \item  В каждой внутренней вершине дерева выдаётся некоторый прогноз
  \item  В каждой листовой вершине дерева проверятеся некоторое условие
  \item  Каждая листова вершина дерева связана как минимум ещё с двумя вершинами
%   \item  Чем глубже дерево, тем более сложная получается модель
%   \item  Чем глубже дерево, тем менее переобученная получается модель
  \item  С помощью решающего дерева можно идеально решить задачи только с линейно разделимой выборкой 
\end{answerlist}
\end{question}

\begin{solution}
\begin{answerlist}
  \item Good answer :)
  \item Bad answer :(
  \item Bad answer :(
  \item Bad answer :(
  \item Bad answer :(
\end{answerlist}
\end{solution}

\begin{question}
Выберите одно верное утверждение про случайный лес
\begin{answerlist}
%   \item Случайный лес \textbf{НЕ} переобучается с ростом числа деревьев
  \item В случайном лесу каждое новое дерево исправляет ошибки предыдущих
  \item Случайный лес переобучается с ростом числа деревьев
  \item При обучении случайного леса используют \textbf{НЕ} глубокие деревья
  \item Случайный лес непригоден для задачи классификации
%   \item При обучении случайного леса используют глубокие деревья
  \item Случайный лес позволяет оценить обобщающую способность без тестовой выборки
\end{answerlist}
\end{question}

\begin{solution}
\begin{answerlist}
  \item Bad answer :( 
  \item Bad answer :( 
  \item Bad answer :(
  \item Bad answer :( 
  \item Good answer :)
\end{answerlist}
\end{solution}

\newpage 

\begin{question}
Маршал смотрит хоккей. Все предыдущие сезоны он записывал количество игр, которые выиграли <<Викинги>>. Также он записывал количество пинт пива, которое он выпил в баре. Теперь Маршал хочет обучить решающее дерево предсказывать число победных игр по выпитым пинтам. В качестве критерия разбиения вершины на две он использует MSE. Какое значение порога будет выбрано для разбиения на первом уровне дерева? 

\begin{table}[h]
    \centering
    \begin{tabular}{>{\bfseries}cccccccc}
        \toprule
         пинты & $4$ & $5$ & $6$ & $7$ & $8$  \\ \midrule
         победы & $2$ & $4$ & $3$ & $5$ & $10$ \\
         \bottomrule
    \end{tabular}
\end{table}

\begin{answerlist}
  \item  $4.5$
  \item  $5.5$
  \item  $6.5$
  \item  $7.5$
  \item  $8.5$
\end{answerlist}
\end{question}

\begin{solution}
\begin{answerlist}
  \item Bad answer :(
  \item Good answer :) % Лень считать какой правильный 
  \item Bad answer :(
  \item Bad answer :(
  \item Bad answer :(
\end{answerlist}
\end{solution}

\begin{question}
Лили обучает алгоритм машинного обучения предсказывать стоимость картин в долларах. Какой из алгоритмов, перечисленных ниже, может выдать отрицательные прогнозы, несмотря на то, что в обучающей выборке нет отрицательных цен? 
\begin{answerlist}
%   \item  Линейная регрессия
  \item  Глубокое решающее дерево
  \item  Неглубокое решающее дерево
  \item  Случайный лес
  \item  Градиентный бустинг
  \item  Метод ближайших соседей
\end{answerlist}
\end{question}

\begin{solution}
\begin{answerlist}
  \item Bad answer :(
  \item Bad answer :(
  \item Bad answer :(
  \item Good answer :)
  \item Bad answer :(
\end{answerlist}
\end{solution}


\begin{question}
Стелла обучает градиентный бустинг. В качестве функции потерь она использует $MAE$. Как будут выглядеть сдвиги $s_i$? 
\begin{answerlist}
  \item  $y_i - a(x_i)$
  \item  $|y_i - a(x_i)|$
  \item  $sign(y_i - a(x_i))$
  \item  $sign(|y_i - a(x_i)|)$
  \item  $(y_i - a(x_i))^2$
\end{answerlist}
\end{question}

\begin{solution}
\begin{answerlist}
  \item Bad answer :(
  \item Bad answer :(
  \item Good answer :)
  \item Bad answer :(
  \item Bad answer :(
\end{answerlist}
\end{solution}


\begin{question}
Робин обучила случайный лес из трёх деревьев для регрессии. Они предсказали $4, -2, 7.$ Каким будет итоговое предсказание леса? 
\begin{answerlist}
  \item  $3$
  \item  $11$
  \item  $4$
  \item  $6$
  \item  $-3$
\end{answerlist}
\end{question}

\begin{solution}
\begin{answerlist}
  \item Good answer :)
  \item Bad answer :(
  \item Bad answer :(
  \item Bad answer :(
  \item Bad answer :(
\end{answerlist}
\end{solution}


\begin{question}
Выберите одно верное утверждения про смещение и разброс
\begin{answerlist}
  \item У глубоких деревьев высокий разброс и низкое смещение
%   \item Алгоритмы с маленьким смещением больше подвержены переобучению
  \item У деревьев не бывает смещения и разброса
  \item Алгоритмы с маленьким смещением не подвержены переобучению
  \item У глубоких деревьев высокое смещение и низкий разброс
  \item Более сложные модели обычно обладают более высоким смещением
%   \item Более сложные модели обычно обладают более низким смещением
%   \item Алгоритмы с высоким смещением больше подвержены переобучению
\end{answerlist}
\end{question}

\begin{solution}
\begin{answerlist}
  \item Good answer :)
  \item Bad answer :(
  \item Bad answer :(
  \item Bad answer :(
  \item Bad answer :(
\end{answerlist}
\end{solution}

\newpage 

\begin{question}
Выберите одно верное утверждения про K-means:
\begin{answerlist}
  \item  Метод сам выбирает необходимое число кластеров.
  \item  Метод зависит от выбора начального положения центров кластеров
  \item  Метод гарантированно сходится за 1000 итераций
%   \item  Мы должны заказать число кластеров перед запуском метода
  \item  K-means и DBSCAN это один и тот же метод
  \item  Метод находит шумовые объекты и выбросы и не учитывает их при кластеризации
\end{answerlist}
\end{question}

\begin{solution}
\begin{answerlist}
  \item Bad answer :(
  \item Good answer :)
  \item Bad answer :(
  \item Good answer :)
  \item Bad answer :(
\end{answerlist}
\end{solution}


\begin{question}
Тэд Мозби, архитектор, хочет выбрать место для своего нового здания. Каждое место в городе описывается $1000$ признаков. Тэд хочет сжать пространство этих признаков до двух и изобразить все места на плоскости. Какой алгоритм может ему в этом помочь? 
\begin{answerlist}
  \item  Градиентный бустинг
  \item  ALS
  \item  Случайный лес
  \item  DBSCAN
  \item  Логистическая регрессия
\end{answerlist}
\end{question}

\begin{solution}
\begin{answerlist}
  \item Bad answer :(
  \item Bad answer :(
  \item Bad answer :(
  \item Bad answer :(
  \item Bad answer :(
\end{answerlist}
\end{solution}


\begin{question}
Лили рекомендует Тэду заказать себе бургер. Маршал пробовал бургер, ему понравилось. А ещё им обоим до этого понравилась пицца. Как называется такой тип рекомендаций? 
\begin{answerlist}
  \item  Матричная факторизация
  \item  Item based
  \item  ALS
  \item  Косинусные
  \item  User based
\end{answerlist}
\end{question}

\begin{solution}
\begin{answerlist}
  \item Bad answer :(
  \item Bad answer :(
  \item Bad answer :(
  \item Bad answer :(
  \item Good answer :)
\end{answerlist}
\end{solution}

\newpage 

\section*{Часть вторая: открытые вопросы}

Эта часть состоит из открытых вопросов. На них необходимо дать краткие, но ёмкие ответы. За каждый ответ вы можете получить 10 баллов.


\begin{question}
    У Маршала есть диаграмма рассеяния. Постройте по ней классификационное дерево для зависимой переменной $y$:
    
    \begin{center}
      \begin{tikzpicture}[scale = 0.015]
      \input{tree_scatter_data.tikz}
      \end{tikzpicture}
    \end{center}
\end{question}


\vspace{5cm} 


\begin{question}
При обучении случайного леса можно оценивать важность признаков. Опишите здесь один из алгоритмов, с помощью которого это можно сделать.
\end{question}

\newpage


\begin{question}
Виктория рассматривает следующий способ обучения базовой модели $b_N(x)$ в градиентном бустинге для функции потерь $L(y, x)$: 

\[
\frac{1}{\ell} \sum_{i = 1}^{\ell} (s_i - a_N(x_i)) \to \min_{a_N}; \qquad s_i = \frac{\partial}{\partial z} L(s_i, z) \left|_{z = b_{N-1}(x_i)} \right.
\]

Найдите в формулах все ошибки. Объясните, почему это ошибки и исправьте их. 
\end{question}


\vspace{7cm} 

\begin{question}
Объясните, как работает метод главных компонент. Для решения каких задач он может использоваться? 
\end{question}

\newpage 

\begin{question}
Братаны собрались на площади и собираются разбиться на группы с помощью алгоритма DBSCAN. В качестве параметра $\varepsilon$ братаны использут радиус синей окружности, нанесённой на рисунок. Параметр \textit{min\_samples} равен $3$. Сам братан, в окрестности которого мы ищем других братанов, тоже входит в эту тройку. 

Сколько кластеров и почему выделит DBSCAN? Отметьте их на картинке. Приведите пример объектов, которые окажутся граничными. Отметьте их на рисунке и объясните почему так произойдёт. 

    \begin{center}
        \includegraphics[scale=0.25]{bro_DBSCAN.png}
    \end{center}
\end{question}


\newpage 


\section*{Часть третья: задачки}

Решите все задания. Все ответы должны быть обоснованы. Решения должны быть прописаны для каждого пункта. Рисунки должны быть чёткими и понятными. Все линии должны быть подписаны. За решение каждой задачи вы можете получить 10 баллов.


\begin{question}
На картинке ниже синими точками изображены бары. Ребята хотят кластеризовать их с помощью $K$-means на $4$ кластера. Помогите им.

Для поиска расстояний используется манхеттенская метрика. В качестве начальных точек используются $(0, 2)$, $(0, 1)$, $(−1, −3)$, $(−1, −2)$.

\begin{center}
    \includegraphics[scale=0.3]{for_k_means.png}
\end{center}
\end{question}


\newpage 

\begin{question}
Барни и Тэд зашли на youtube со своих компьютеров. Они получили одинаковую выдачу из четырёх видосов: 

\begin{enumerate}
    \item  History of the empire state building
    \item  Robin Sparkles-Let's Go To The Mall'
    \item  How to use playbook correctly
    \item  Barney Stinson - Nothing Suits Me Like A Suit
\end{enumerate}

Для Барни оказались релевантны второй и третий ролилки. Для Тэда первый и четвёртый. Посчитайте значение метрики $map@4.$ Объясните в чём заключается её смысл. 
\end{question}


\end{document}

