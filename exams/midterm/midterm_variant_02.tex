\documentclass[12pt]{article}

\usepackage{tikz} % картинки в tikz
\usepackage{microtype} % свешивание пунктуации
\usepackage{array} % для столбцов фиксированной ширины
\usepackage{comment} % для комментирования целых окружений
\usepackage{indentfirst} % отступ в первом параграфе

\usepackage{sectsty} % для центрирования названий частей
\allsectionsfont{\centering}

\usepackage{amsmath, amssymb, amsthm, amsfonts} % куча стандартных математических плюшек

\usepackage[top=2cm, left=1cm, right=1cm, bottom=2cm]{geometry} % размер текста на странице
\usepackage{lastpage} % чтобы узнать номер последней страницы
 
\usepackage{enumitem} % дополнительные плюшки для списков
%  например \begin{enumerate}[resume] позволяет продолжить нумерацию в новом списке

\usepackage{caption} % подписи к рисункам
\usepackage{hyperref} % гиперссылки
\usepackage{multicol} % текст в несколько столбцов


\usepackage{fancyhdr} % весёлые колонтитулы
\pagestyle{fancy}
\lhead{Введение в машинное обучение, ВШЭ}
\chead{}
\rhead{2022-04-01}
\lfoot{Вариант $\Omega \Gamma \Delta$}
\rfoot{Паниковать запрещается!}
%\rfoot{Тест}
\renewcommand{\headrulewidth}{0.4pt}
\renewcommand{\footrulewidth}{0.4pt}

\usepackage{ifthen} % для написания условий

\usepackage{todonotes} % для вставки в документ заметок о том, что осталось сделать
% \todo{Здесь надо коэффициенты исправить}
% \missingfigure{Здесь будет Последний день Помпеи}
% \listoftodos --- печатает все поставленные \todo'шки


% более красивые таблицы
\usepackage{booktabs}
% заповеди из докупентации:
% 1. Не используйте вертикальные линни
% 2. Не используйте двойные линии
% 3. Единицы измерения - в шапку таблицы
% 4. Не сокращайте .1 вместо 0.1
% 5. Повторяющееся значение повторяйте, а не говорите "то же"


\usepackage{fontspec}
\usepackage{polyglossia}

\setmainlanguage{russian}
\setotherlanguages{english}

% download "Linux Libertine" fonts:
% http://www.linuxlibertine.org/index.php?id=91&L=1
\setmainfont{Linux Libertine O} % or Helvetica, Arial, Cambria
% why do we need \newfontfamily:
% http://tex.stackexchange.com/questions/91507/
\newfontfamily{\cyrillicfonttt}{Linux Libertine O}

% Математические шрифты 
% Математические шрифты 
\usepackage{unicode-math}     
\setmathfont[math-style=upright]{euler.otf} 

\setmathfont[range={\mathbb, \mathop, \heartsuit, \angle, \smile, \varheartsuit}]{Asana-Math.otf}

\AddEnumerateCounter{\asbuk}{\russian@alph}{щ} % для списков с русскими буквами
\setlist[enumerate, 2]{label=\asbuk*),ref=\asbuk*}


% мои цвета https://www.artlebedev.ru/colors/
\definecolor{titleblue}{rgb}{0.2,0.4,0.6} 
\definecolor{blue}{rgb}{0.2,0.4,0.6} 
\definecolor{red}{rgb}{1,0,0.2} 
\definecolor{green}{rgb}{0,0.6,0} 
\definecolor{purp}{rgb}{0.4,0,0.8} 

% цвета из geogebra 
\definecolor{litebrown}{rgb}{0.6,0.2,0}
\definecolor{darkbrown}{rgb}{0.75,0.75,0.75}

% Гиперссылки
\usepackage{xcolor}   % разные цвета

\usepackage{hyperref}
\hypersetup{
  unicode=true,           % позволяет использовать юникодные символы
  colorlinks=true,        % true - цветные ссылки
  urlcolor=blue,          % цвет ссылки на url
  linkcolor=black,          % внутренние ссылки
  citecolor=green,        % на библиографию
  breaklinks              % если ссылка не умещается в одну строку, разбивать её на две части?
}

% эпиграфы
\usepackage{epigraph}
\setlength\epigraphwidth{.7\textwidth}
\setlength\epigraphrule{0pt}

% Математические операторы первой необходимости:
\DeclareMathOperator{\sgn}{sign}
\DeclareMathOperator*{\argmin}{arg\,min}
\DeclareMathOperator*{\argmax}{arg\,max}
\DeclareMathOperator{\Cov}{Cov}
\DeclareMathOperator{\Var}{Var}
\DeclareMathOperator{\Corr}{Corr}
\DeclareMathOperator{\E}{\mathop{E}}
\DeclareMathOperator{\Med}{Med}
\DeclareMathOperator{\Mod}{Mod}
\DeclareMathOperator*{\plim}{plim}

\DeclareMathOperator{\logloss}{logloss}
\DeclareMathOperator{\softmax}{softmax}

\DeclareMathOperator{\tr}{tr}

% команды пореже
\newcommand{\const}{\mathrm{const}}  % const прямым начертанием
\newcommand{\iid}{\sim i.\,i.\,d.}  % ну вы поняли...
\newcommand{\fr}[2]{\ensuremath{^{#1}/_{#2}}}   % особая дробь
\newcommand{\ind}[1]{\mathbbm{1}_{\{#1\}}} % Индикатор события
\newcommand{\dx}[1]{\,\mathrm{d}#1} % для интеграла: маленький отступ и прямая d

% одеваем шапки на частые штуки
\def \hb{\hat{\beta}}
\def \hs{\hat{s}}
\def \hy{\hat{y}}
\def \hY{\hat{Y}}
\def \he{\hat{\varepsilon}}
\def \hVar{\widehat{\Var}}
\def \hCorr{\widehat{\Corr}}
\def \hCov{\widehat{\Cov}}

% Греческие буквы
\def \a{\alpha}
\def \b{\beta}
\def \t{\tau}
\def \dt{\delta}
\def \e{\varepsilon}
\def \ga{\gamma}
\def \kp{\varkappa}
\def \la{\lambda}
\def \sg{\sigma}
\def \tt{\theta}
\def \Dt{\Delta}
\def \La{\Lambda}
\def \Sg{\Sigma}
\def \Tt{\Theta}
\def \Om{\Omega}
\def \om{\omega}

% Готика
\def \mA{\mathcal{A}}
\def \mB{\mathcal{B}}
\def \mC{\mathcal{C}}
\def \mE{\mathcal{E}}
\def \mF{\mathcal{F}}
\def \mH{\mathcal{H}}
\def \mL{\mathcal{L}}
\def \mN{\mathcal{N}}
\def \mU{\mathcal{U}}
\def \mV{\mathcal{V}}
\def \mW{\mathcal{W}}

% Жирные буквы
\def \mbb{\mathbb}
\def \RR{\mbb R}
\def \NN{\mbb N}
\def \ZZ{\mbb Z}
\def \PP{\mbb{P}}
\def \QQ{\mbb Q}

\def \putyourname{\fbox{
    \begin{minipage}{42em}
      Фамилия, имя, номер группы:\vspace*{3ex}\par
      \noindent\dotfill\vspace{2mm}
    \end{minipage}
  }
}

\def \checktable{

  \vspace{5pt}
  Табличка для проверяющих работу:

\vspace{5pt}

  \begin{tabular}{|m{2cm}|m{1cm}|m{1cm}|m{1cm}|m{1cm}|m{1cm}|m{2cm}|}
\toprule
    Тест & 1 &  2 & 3 & 4 & 5 & Итого \\
\midrule
    &  &  & & & & \\
    &  &  & & & & \\
 \bottomrule
\end{tabular}
}


\def \testtable{

\vspace{5pt}
  Внесите сюда ответы на тест:

\vspace{5pt}

\begin{tabular}{|m{2cm}|m{0.6cm}|m{0.6cm}|m{0.6cm}|m{0.6cm}|m{0.6cm}|m{0.6cm}|m{0.6cm}|m{0.6cm}|m{0.6cm}|m{0.6cm}|}
\toprule
    Вопрос & 1 &  2 & 3 & 4 & 5 & 6 & 7 & 8 & 9 & 10 \\
\midrule
    Ответ &  &  & & & & & & & & \\
 \bottomrule
\end{tabular}
}


% [1][3] 1 = one argument, 3 = value if missing
% эта магия создаёт окружение answerlist
% именно в окружении answerlist записаны варианты ответов в подключаемых exerciseXX
% просто \begin{answerlist} сделает ответы в три столбца
% если ответы длинные, то надо в них руками сделать
% \begin{answerlist}[1] чтобы они шли в один столбец
\newenvironment{answerlist}[1][3]{
\begin{multicols}{#1}

\begin{enumerate}[label=\fbox{\emph{\Alph*}},ref=\emph{\alph*}]
}
{
\item Нет верного ответа.
\end{enumerate}
\end{multicols}
}

% BB: unicol version. don't know why \ifthenelse fails in second part of new-env
\newenvironment{answerlistu}{
\begin{enumerate}[label=\fbox{\emph{\Alph*}},ref=\emph{\alph*}]
}
{
\item Нет верного ответа.
\end{enumerate}
}


\excludecomment{solution} % without solutions

\theoremstyle{definition}
\newtheorem{question}{Вопрос}

\usepackage{tikzlings}
\usepackage{tikzducks}

\usepackage{alltt}

\usepackage{emoji}

\begin{document}

\putyourname

\testtable

\checktable

\epigraph{Позитивная мотивация — явно не мой конёк, и мы все умрём. \\ Всё уже было до нас, можно выдохнуть страх, и уставить глаза в небосклон. \\ Если этой контрольной и сопротивляться, то не с печальным лицом. \\ Всё повторится не раз, но мы живы сейчас, нами рано удобрять чернозём.}{\textit{Oxxxymiron и Тося Чайкина про мидтёрм по ML (2022)}}


Работа состоит из трёх частей: тестовая, задачи и ответы на открытые вопросы. Списывание карается обнулением работы. Удачи!


\section*{Часть первая: тестовая} 

Дайте ответ на $10$ тестовых вопросов. Каждый вопрос стоит $3$ балла. Никакие дополнительные пояснений в этой части работы от вас не требуются.

\begin{question}
У императора Куско есть база данных обо всех бизонах, пасущихся на просторах его необъятной империи. Для удобства император хочет разбить всех бизонов на $30$ стад таким образом, чтобы в каждом были похожие по своим характеристикам особи. Какую задачу решает император солнца? 
\begin{answerlist}
   \item  Регрессия
   \item  Классификация
   \item  Кластеризация
   \item  Ранжирование
   \item  Рекомендаии
\end{answerlist}
\end{question}

\begin{solution}
\begin{answerlist}
  \item Bad answer :(
  \item Good answer :)
  \item Bad answer :(
  \item Bad answer :(
  \item Bad answer :(
\end{answerlist}
\end{solution}


\begin{question}
Для чего можно использовать кросс-валидацию?
\begin{answerlist}
   \item  Для подбора лучшего коэффициента регуляризации
   \item  Для оценки того, как модель будет работать на новых данных
   \item  Для подбора оптимального способа измерения ошибки модели на новых данных
  \item  Для выбора лучшего типа модели
  \item  Для улучшения качества обучающей выборки
\end{answerlist}
\end{question}

\begin{solution}
\begin{answerlist}
  \item Good answer :)
  \item Bad answer :(
  \item Bad answer :(
  \item Good answer :)
  \item Good answer :)
\end{answerlist}
\end{solution}

\newpage 

\begin{question}
Выберите все верные утверждения про переобучение (overfitting).
\begin{answerlist}
   \item  Метод ближайших соседей на этапе обучения просто запоминает всю выборку, поэтому он не переобучается
   \item  $L_2$-регуляризатор добавляют в линейную модель, чтобы не дать ей переобучиться
   \item  Если качество модели на тестовой выборке ниже, чем на обучающей, скорее всего, модель переобучилась
   \item  Валидационную выборку выделяют, чтобы подобрать на ней гиперпараметры
   \item  Чтобы избежать переобучения, качество модели достаточно измерить на той же самой выборке, на которой обучаешь модель
\end{answerlist}
\end{question}

\begin{solution}
\begin{answerlist}
  \item Bad answer :(
  \item Good answer :)
  \item Good answer :)
  \item Good answer :)
  \item Bad answer :(
\end{answerlist}
\end{solution}


\begin{question}
Выберите все верные утверждения про градиентный спуск
\begin{answerlist}
   \item  Градиентный спуск гарантированно находит глобальный оптимум
   \item  Метод ближайших соседей обучается градиентным спуском
   \item  Градиентный спуск не гарантирует нахождение глобального оптимума
   \item  Масштабирование признаков ускоряет работу градиентного спуска
   \item  При обучении логистической регрессии с помощью градиентного спуска напрямую оптимизируют долю верных ответов, $accuracy$
\end{answerlist}
\end{question}

\begin{solution}
\begin{answerlist}
  \item Good answer :)
  \item Bad answer :(
  \item Bad answer :(
  \item Good answer :)
  \item Good answer :)
\end{answerlist}
\end{solution}


\begin{question}
Какие из сопособов приведённых ниже можно использовать для борьбы с выбросами при обучении линейных моделей? 
\begin{answerlist}
   \item Заменить выбросы на какой-нибудь квантиль, например, медиану
   \item Выбросить все наблюдения-выбросы
   \item Использовать функцию потерь, которая нечувствительна к выбросам (Huber Loss и тп)
   \item Стандартизировать данные
   \item Выбросы --- это категориальные переменные, для них можно сделать OHE-преобразование
\end{answerlist}
\end{question}

\begin{solution}
\begin{answerlist}
  \item Good answer :)
  \item Good answer :)
  \item Good answer :)
  \item Bad answer :(
  \item Bad answer :(
\end{answerlist}
\end{solution}


\begin{question}
Рассмотрим выборку, состоящую из одного признака, и линейную модель над этой выборкой. Говорят, что эту модель можно изобразить как прямую. А в каких координатах? Признак обозначается $x$, целевая переменная $y$, веса модели — $w_0$ и $w_1$
\begin{answerlist}
   \item  $x$
   \item  $(x, y)$
   \item  $(w_1, y)$
   \item  $(w_0, w_1)$
   \item  $(x, y, w_0, w_1)$
\end{answerlist}
\end{question}

\begin{solution}
\begin{answerlist}
  \item Bad answer :(
  \item Good answer :)
  \item Bad answer :(
  \item Bad answer :(
  \item Bad answer :(
\end{answerlist}
\end{solution}

\begin{question}
Какие из функций ниже логично использовать для оценки качества линейной модели на тестовой выборке? 
\begin{answerlist}
   \item  \( \frac{1}{n} \sum_{i=1}^n (y_i - \hat y_i) \)
   \item  \( \frac{1}{n} \sum_{i=1}^n|y_i - \hat y_i| \)
   \item  \( \frac{1}{n} \sum_{i=1}^n (y_i - \hat y_i)^2 + \sum_{j=1}^k w_j^2 \)
   \item  \( \frac{1}{n} \sum_{i=1}^n |y_i - \hat y_i| + \sum_{j=1}^k w_j^2 \)
   \item  \( \sum_{j=1}^k w_j^2 \)
\end{answerlist}
\end{question}

\begin{solution}
\begin{answerlist}
  \item Bad answer :(
  \item Good answer :)
  \item Bad answer :(
  \item Bad answer :(
  \item Bad answer :(
\end{answerlist}
\end{solution}

\newpage 

\begin{question}
Льюис и Кларк идут с экспедицией на Дикий Запад и предсказывают население каждого нового племени с помощью метода двух ближайших соседей. Все прогнозы строятся с учётом того, каким получилось расстояние между племенами. 

Если нарисовать регион на карте, окажется что племя Сиу чилсенностью $10$ тыc. живёт по координатам $(0, 0)$. Племя Ото численностью $20$ тыс. живет по координатам $(3, 5)$.  Пляемя Миссури живёт по координатам  $(0, 1)$. Каким будет примерный прогноз для его численности?  

\begin{answerlist}
   \item $12$ тыс.
   \item $15$ тыс.
   \item $20$ тыс.
   \item $10$ тыс.
   \item $14$ тыс.
\end{answerlist}
\end{question}

\begin{solution}
\begin{answerlist}
  \item Good answer :)
  \item Bad answer :(
  \item Bad answer :(
  \item Bad answer :(
  \item Bad answer :(
\end{answerlist}
\end{solution}


\begin{question}
Мир пришёл на Великие Равнины. Несколько индейских племён зарыли топор войны и решили раскочегарить трубку мира. Шаман племени Апачи умеет предсказывать, будет ли заключён мирный договор. Он делает с помощью логистической регрессии, обученной на предыдущих конфликтах: 

\[ 
\mathbb{P}(y_i = 1 \mid x_i) =  \frac{1}{1 + exp(100 - 20 \cdot x_i)},
\]

где $x$ --- время в минутах, которое вожди провели за трубкой мира.  Выберите все верные утверждения об этой модели. 

\begin{answerlist}
   \item  Если трубку мира не раскурят вообще, мир наступит с очень низкой вероятностью
   \item  Если вожди курили трубку $5$ минут, мир будет заключен с вероятностью $0.25$
   \item  С каждой дополнительной минутой за трубкой мира, вероятность заключить мир уменьшается 
   \item  \emoji{green-apple} \emoji{red-apple}  \emoji{dashing-away} \mbox{ } \emoji{smiling-face-with-sunglasses} \emoji{thumbs-up}
   \item  С каждой дополнительной минутой за трубкой мира, вероятность заключить мир растёт
\end{answerlist}
\end{question}

\begin{solution}
\begin{answerlist}
  \item Good answer :)
  \item Bad answer :(
  \item Bad answer :(
  \item Good answer :)
  \item Good answer :)
\end{answerlist}
\end{solution}




\begin{question}
Мы хотим обучать модель, напрямую максимизируя полноту. Выберите в списке ниже все причины, почему это плохая идея? 
\begin{answerlist}
   \item Максимальную полноту можно получить тривиальной моделью, которая все объекты относит к положительному классу
   \item Полнота зависит от порога
   \item Полноту нельзя продиффериенцировать 
   \item Полнота очень плохо работает, если выборка несбалансированная
   \item Максимальную полноту можно получить тривиальной моделью, которая все объекты относит к отрицательному классу
\end{answerlist}
\end{question}

\begin{solution}
\begin{answerlist}
  \item Good answer :)
  \item Good answer :)
  \item Good answer :)
  \item Bad answer :(
  \item Bad answer :(
\end{answerlist}
\end{solution}




\newpage 

 
\section*{Часть вторая: открытые вопросы}

Эта часть состоит из открытых вопросов. На них необходимо дать краткие, но ёмкие ответы. За каждый ответ вы можете получить 10 баллов.

\begin{question}
Предложите для каждой из перечисленных ниже задач, как сформулировать их в терминах машинного обучения: укажите, что будет являться объектом и целевой переменной, а также напишите тип задачи.

\begin{enumerate}
    \item Фармкомпания Wachowski Inc. производит два вида таблеток: красные и синие. К сожалению, конвейерная лента на заводе иногда выходит из строя и в одну блистерную упаковку попадают таблетки разного цвета. Мы хотели бы отбраковывать такие упаковки.
    
  \item Студент решил воспользоваться высокими технологиями при подготовке к экзамену и строит модель, которая предскажет его оценку.
\end{enumerate}
\end{question}


\vspace{4cm} 


\begin{question}
Несколько архитекторов строят мемориал в честь вождя племени Оглала, Неистового Коня. Архитекторы хотят оптимизировать поставки стройматериалов. Для этого они предсказывают время прибытия поставки по разным факторам: трафик на дорогах, погода, загруженность складов и т.п. 

Архитекторы хотят понять, какие факторы влияют на доставку сильнее всего. Они обучают линейную регрессию и хотят занулить переменные перед всеми факторами, которые несущественны для их прогнозов. Используется следуюшая функция потерь:

$$
Q(w) = \frac{1}{n} \sum_{i=1}^n (a(x_i) - y_i) + \lambda \cdot \sum_{i=1}^d w_i^2 \to \max_{w}
$$

Какие ошибки вы тут видите? Для каждой объясните, к каким последствиям и почему она приведёт, а также как это исправить.
\end{question}



\newpage 

\begin{question}
Объясните мем
\begin{center}
  \includegraphics[scale=0.3]{memes3.png}
\end{center}
\end{question}

\vspace{5cm} 


\begin{question}
Линейные модели в машинном обучении обучаются с помощью градиентного спуска. На каждой итерации вектор весов $w$ изменяется по какой-то формуле. Выпишите эту формулу. Объясните в ней каждую компоненту и обозначение. 
\end{question}


\newpage 

\begin{question}
Вождь Маквагабо\footnote{Есть теория, что Голландцам землю за бусы продали не местные индейцы, а шайка мошенников, которая даже не владела этой землёй.} хочет классифицировать бусы из бисера. У него есть $1000$ синих бус и $10^6$ красных. 

Почему для оценки качества классификации не получится использовать долю правильных ответов, accuracy? Приведите примеры моделей, где эта метрика показывает неадекватный результат. Что можно сделать для корректной оценки работы модели в таком случае? 
\end{question}


\newpage 


\section*{Часть третья: задачки}

Решите все задания. Все ответы должны быть обоснованы. Решения должны быть прописаны для каждого пункта. Рисунки должны быть чёткими и понятными. Все линии должны быть подписаны. За решение каждой задачи вы можете получить 10 баллов.

\begin{question}
Индейские Шаманы предсказывают стоимость недвижимости в Сиэтле.  Шаман Одэхингум (лёгкое колебание воды) использует метод ближайших соседей. Шаман Пэпина (виноградная лоза, растущая вокруг дуба) использует линейную регрессию. Шаман Апониви (где ветер вырывает промежуток с корнем) использует случайный лес. 

В тестовой выборке у них есть три дома $y_i$. Для каждого из них шаманы построили прогнозы. 

\begin{center}
  \begin{tabular}{c|c|c|c}
    настоящие $y_i$ &  4 & 7 & 1 \\
    \hline
    KNN & 4 & 5 & 2  \\
    линейная регрессия &  5 & 6 & 0 \\
    случайный лес & 1 & 1 & 1 \\
  \end{tabular}
\end{center}

\begin{enumerate}
    \item Найдите для прогнозов $MAE$, $MSE$, $RMSE$ и $MAPE$.
    \item Объясните, зачем от $MSE$ обычно переходят к $RMSE$.
    \item Объясните, почему $MAE$ считается более устойчивой к выбросам.
\end{enumerate}
\end{question}

\newpage 

\begin{question}
    Покахонтас спасла от смерти капитана Джона Смита, а затем решила задачу классификации. У неё получились следующие прогнозы. 
    
    \begin{center}
      \begin{tabular}{c|c}
        $y_i$ & $\hat p_i$ \\
        \hline
        $1$  & $0.9$ \\
        $0$ & $0.1$ \\
        $0$ & $0.75$ \\
        $0$ & $0.56$ \\
        $0$ & $0.2$ \\
        $1$ & $0.37$ \\
        $0$ & $0.25$ \\   
      \end{tabular}
    \end{center}
    
    \begin{enumerate}
      \item  Бинаризуйте ответ по порогу $t$ и посчитайте точность и полноту для $t = 0.3$ и для  $t = 0.8$.
      \item Постройте ROC-кривую и найдите площадь под ней. 
    \end{enumerate}
\end{question}

\end{document}